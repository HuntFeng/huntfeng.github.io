\documentclass{article}

\usepackage{amsmath}

\usepackage{amssymb}

\usepackage{amsthm}

\usepackage{graphicx}



\newtheorem{definition}{Definition}

\newtheorem{property}{Property}

\newtheorem{ex}{Example}

\newtheorem{thm}{Theorem}

\newtheorem{lemma}{Lemma}

\newtheorem{prop}{Proposition}



\begin{document}
\begin{definition}
Tautology is a compound statement that is true for all circumstances. In other words, the truth value of a tautology is always True.
\end{definition}

\begin{ex}
	The statement "I am either dead or alive" is a tautology. Since this statement is always true. Its expression can be written as 
	\[ p\vee \neg p \]
	where p=I am dead. The truth value table of this formula is as shown below
	
	
	\begin{center}
		\begin{tabular}{|c|c|c|}
		\hline 
		$p$ & $\neg p$ & $p\vee \neg p$  \\ 
		\hline 
		0 & 1 & 1 \\ 
		\hline 
		1 & 0 & 1 \\ 
		\hline 
	\end{tabular}
	\end{center} 
\end{ex}

\begin{definition}
	An \textbf{argument} is a series of statements(\textbf{premises}) and ending with a \textbf{conclusion}.
\end{definition}

\begin{definition}
	An argument \textbf{valid} if and only if when all the premises are true, the conclusion is necessarily true. In other words, an argument is valid if and only if it is equivalent to a tautology.
	
	An argument is not valid if and only if it is not equivalent to a tautology.    
\end{definition}

\begin{thm}
	Logic inference rules
	
	\begin{center}
		\includegraphics*[width=10cm]{inference_rule.PNG}
	\end{center}
	
	\iffalse
	\begin{tabular}{|c|c|c|}
		\hline 
		Modus Ponens & Modus Tollens & Disjunctive Syllogism \\
		$p\to q$ & $p\to q$ & $p\vee q$ \vline $p\vee q$ \\
		$p$ 	 & $\neg q$ & $\neg q$  \vline $\neg p$\\
		$\therefore q$ & $\therefore \neg p$ & $\therefore p $\vline $\therefore q$\\
		\hline 
		Addition & Simplification & Proof by contradiction \\
		$p$ \vline $q$ & $p\wedge q$ \vline $p\wedge q$ & $\neg p\to F$\\
		$\therefore p\vee q$\vline $\therefore p\vee q$ & $\therefore p$ \vline $\therefore q$ & $\therefore p$\\
		\hline 
		Hypothetical Syllogism & Conjunction & Proof by divided into two cases \\ 
		$p\to q$ & $p$ & $p\vee q$\\
		$q\to r$ & $q$ & $p\to r$\\
		$\therefore p\to r$ & $\therefore p\wedge q$ &  $q\to r$\\
		& & $\therefore r$ \\
		\hline 
		Resolution & & \\
		$p\vee q$ & & \\
		$\neg p \vee r$ & & \\
		$\therefore q\vee r$  called \textbf{resolvent}& & \\
		\hline
	\end{tabular} 
	\fi
\end{thm}



\end{document}

