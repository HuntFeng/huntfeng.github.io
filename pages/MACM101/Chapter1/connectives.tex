\documentclass{article}

\usepackage{amsmath}

\usepackage{amssymb}

\usepackage{amsthm}

\usepackage{graphicx}



\newtheorem{definition}{Definition}

\newtheorem{property}{Property}

\newtheorem{ex}{Example}

\newtheorem{thm}{Theorem}

\newtheorem{lemma}{Lemma}

\newtheorem{prop}{Proposition}



\begin{document}
\begin{definition}
	Compound propositions are made of primitive propositions.
\end{definition}

\begin{definition}
	Logic connectives and their symbols are
	\begin{center}
			\begin{tabular}{|c|c|c|}
			\hline 
			negation& not & $\neg$ \\ 
			\hline 
			conjunction& and & $\wedge$ \\ 
			\hline 
			disjunction& or & $\vee$ \\ 
			\hline 
			exclusive OR& either...or... & $\oplus$ \\ 
			\hline 
			implication& if...then... & $\to$ \\ 
			\hline 
			equivalent& if and only if  & $\leftrightarrow$ \\ 
			\hline 
		\end{tabular} 
	\end{center}
	
	\begin{tabular}{|c|c|}
		\hline 
		$p$ & $\neg p$ \\ 
		\hline 
		0 & 1 \\ 
		\hline 
		1 & 0 \\ 
		\hline 
	\end{tabular} 
	\quad
	\begin{tabular}{|c|c|c|}
		\hline 
		$p$& $q$ & $p\wedge q$ \\ 
		\hline 
		1 & 0 & 0 \\ 
		\hline 
		1 & 1 & 1 \\ 
		\hline 
		0 & 0 & 0 \\ 
		\hline 
		0 & 1 & 0 \\ 
		\hline
	\end{tabular} 
	\quad
	\begin{tabular}{|c|c|c|}
		\hline 
		$p$ & $q$ & $p\vee q$ \\ 
		\hline 
		1 & 0 & 1 \\ 
		\hline 
		1 & 1 & 1 \\ 
		\hline 
		0 & 0 & 0 \\ 
		\hline 
		0 & 1 & 1 \\ 
		\hline 
	\end{tabular}

	\begin{tabular}{|c|c|c|}
		\hline 
		$p$ & $q$ & $q\oplus q$ \\ 
		\hline 
		1 & 0 & 1 \\ 
		\hline 
		1 & 1 & 0 \\ 
		\hline 
		0 & 0 & 0 \\ 
		\hline 
		0 & 1 & 1 \\ 
		\hline 
	\end{tabular}
	\quad
	\begin{tabular}{|c|c|c|}
		\hline 
		$p$ & $q$ & $p\to q$ \\ 
		\hline 
		1 & 0 & 0 \\ 
		\hline 
		1 & 1 & 1 \\ 
		\hline 
		0 & 0 & 1 \\ 
		\hline 
		0 & 1 & 1 \\ 
		\hline 
	\end{tabular} 
	\quad
	\begin{tabular}{|c|c|c|}
		\hline 
		$p$ & $q$ & $p\leftrightarrow q$ \\ 
		\hline 
		1 & 0 & 0 \\ 
		\hline 
		1 & 1 & 1 \\ 
		\hline 
		0 & 0 & 0 \\ 
		\hline 
		0 & 1 & 0 \\ 
		\hline 
	\end{tabular}   

\end{definition}

\begin{definition}
	Two statements are logically equivalent to each other if they have the same truth table. 
\end{definition}

\begin{thm}
	\textbf{Law of substitution}: If two statements are logically equivalent, one can substitute the other.
\end{thm}


\begin{definition}
	Logic substitution rules
	
	\begin{tabular}{|c|c|c|}
		\hline 
		Commtative & $p\wedge q \Leftrightarrow q\wedge p $ & $p\vee q \Leftrightarrow q\vee p $ \\ 
		\hline 
		Associative & $(p\wedge q)\wedge r \Leftrightarrow q\wedge (p\wedge r) $ & $(p\vee q)\vee r \Leftrightarrow q\vee (p\vee r) $  \\ 
		\hline 
		Distributive & $p\wedge (q\vee r) \Leftrightarrow (p\wedge q) \vee (p\wedge r) $ & $p\vee (q\wedge r) \Leftrightarrow (p\vee q) \wedge (p\vee r) $ \\ 
		\hline 
		Identity & $p\wedge T \Leftrightarrow p$ & $p\vee F \Leftrightarrow F$ \\ 
		\hline 
		Negation & $p\vee \neg p \Leftrightarrow T$ & $p\wedge \neg p \Leftrightarrow F$ \\ 
		\hline 
		Double Negation & $\neg(\neg p) \Leftrightarrow p$ &  \\ 
		\hline 
		Idempotent & $p\wedge p \Leftrightarrow p$ & $p\vee p \Leftrightarrow p$ \\ 
		\hline 
		Universal Bound & $p\vee T\Leftrightarrow T$ & $p\wedge F \Leftrightarrow F$ \\ 
		\hline 
		De Morgan's & $\neg(p \wedge q) \Leftrightarrow (\neg p) \vee (\neg q)$ & $\neg(p \vee q) \Leftrightarrow (\neg p) \wedge (\neg q)$ \\ 
		\hline 
		Absorption & $p\vee(p\wedge q)\Leftrightarrow p$ & $p\wedge (p \vee q) \Leftrightarrow p$ \\ 
		\hline 
		Conditional & $(p\to q) \Leftrightarrow (\neg p \vee q) $ & $\neg (p\to q) \Leftrightarrow (p\wedge \neg q) $ \\
		\hline
	\end{tabular} 
\end{definition}


\begin{definition}
	Logic inference rules
	
	\begin{tabular}{|c|c|c|}
		\hline 
		Modus Ponens & Modus Tollens & Disjunctive Syllogism \\
		$p\to q$ & $p\to q$ & $p\vee q$ \vline $p\vee q$ \\
		$p$ 	 & $\neg q$ & $\neg q$  \vline $\neg p$\\
		$\therefore q$ & $\therefore \neg p$ & $\therefore p $\vline $\therefore q$\\
		\hline 
		Disjunctive Addition & Conjunctive Simplification & Rule of contradiction \\
		$p$ \vline $q$ & $p\wedge q$ \vline $p\wedge q$ & $\neg p\to F$\\
		$\therefore p\vee q$\vline $\therefore p\vee q$ & $\therefore p$ \vline $\therefore q$ & $\therefore p$\\
		\hline 
		Hypothetical Syllogism & Conjunctive Addition & Dilemma \\ 
		$p\to q$ & $p$ & $p\vee q$\\
		$q\to r$ & $q$ & $p\to r$\\
		$\therefore p\to r$ & $\therefore p\wedge q$ &  $q\to r$\\
		 & & $\therefore r$ \\
		\hline 
	\end{tabular} 
\end{definition}



\end{document}

