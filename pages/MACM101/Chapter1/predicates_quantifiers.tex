\documentclass{article}

\usepackage{amsmath}

\usepackage{amssymb}

\usepackage{amsthm}

\usepackage{graphicx}



\newtheorem{definition}{Definition}

\newtheorem{property}{Property}

\newtheorem{ex}{Example}

\newtheorem{thm}{Theorem}

\newtheorem{lemma}{Lemma}

\newtheorem{prop}{Proposition}



\begin{document}
\begin{definition}
	\textbf{Predicate} is a statement that its truth value depends on one or more variables. The set of values that can assign to the variables are called the \textbf{domain}(or \textbf{universe}) of the variable.
\end{definition}

\begin{ex}
	\begin{align}
	&p(x)\equiv\text{x is odd}.\;\;x\in\mathbb{N}\\
	&q(x,y)\equiv x<y\;\;x,y\in\mathbb{R}\\
	&r(x,y,z)\equiv x+y=z\;\;x,y,z\in\mathbb{Z}
	\end{align}
	These are all predicates since their truth values are all depend on the variables.
\end{ex}


\begin{definition}
	The universal quantifier, $\forall$, means: for all, every, any, etc...
\end{definition}

\begin{ex}
	\begin{align}
	&\forall x\in \mathbb{N}, \text{x is an integer}.\\
	&\forall x\in \mathbb{R}, x^2+1 >0
	\end{align}
\end{ex}


\begin{definition}
	The existential quantifier, $\exists$, means: for some, there is, exists, at least one, etc...
\end{definition}

\begin{ex}
	\begin{align}
	&\exists x\in\mathbb{Z} \text{ such that x is even}\\
	&\exists x\in\mathbb{N} \text{ such that } x>5
	\end{align}
\end{ex}

\begin{property}
	\begin{align*}
	\neg (\forall x, p(x)) \Leftrightarrow \exists x\;s.t.\;\neg p(x)\\
	\neg (\exists x\;s.t.\;p(x)) \Leftrightarrow \forall x, \neg p(x)
	\end{align*}
\end{property}


\end{document}

