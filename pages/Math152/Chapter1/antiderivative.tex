\documentclass{article}

\usepackage{amsmath}

\usepackage{amssymb}

\usepackage{amsthm}

\usepackage{graphicx}



\newtheorem{definition}{Definition}

\newtheorem{property}{Property}

\newtheorem{ex}{Example}

\newtheorem{thm}{Theorem}

\newtheorem{lemma}{Lemma}

\newtheorem{prop}{Proposition}



\begin{document}
\begin{definition}
	\textbf{Antiderivative} or \textbf{Indefinite integral} is the reverse operation of derivative. Suppose a function $F(x)$ has derivative $f(x)$, then the antiderivative of $f(x)$ is 
	\[ \int f(x) dx = F(x) + C \] 
	Where $C\in\mathbb{R}$ is a constant.  
\end{definition}

\begin{property}
	\begin{align}
	\int [f(x)\pm g(x)]dx = \int f(x) dx \pm \int g(x) dx\\
	\int kf(x) dx = k\int f(x) dx\;\; (k\in\mathbb{R})\\
	\end{align}
\end{property}


Since the antiderivative is just the reverse operation of derivative, thus we have a set of formulas
\begin{center}
	\includegraphics*[width=10cm]{rules.jpg}
\end{center}



\begin{ex}
	Find the antiderivative of $f(x)= x^3 + \frac{\cos x}{2}$
	\begin{align*}
	\int [x^3 + \frac{\cos x}{2}]dx &= \int x^3 dx + \frac{1}{2}\int \cos x dx\\
	&= \frac{x^4}{4} + \sin x +C
	\end{align*}
\end{ex}



\end{document}

