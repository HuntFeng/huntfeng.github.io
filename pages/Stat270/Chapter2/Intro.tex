\documentclass{article}

\usepackage{amsmath}

\usepackage{amssymb}

\usepackage{amsthm}

\usepackage{graphicx}



\newtheorem{definition}{Definition}

\newtheorem{property}{Property}

\newtheorem{ex}{Example}

\newtheorem{thm}{Theorem}

\newtheorem{lemma}{Lemma}

\newtheorem{prop}{Proposition}



\begin{document}

We often use set theory as the foundation of probability theory. We start by defining the \textbf{sample space} as the set of all possible events.
\begin{ex}
	I have a fair coin, and the sample space for tossing the coin would be
	\[ S = \{ H, T \} \]
	where $H$, and $T$ stands for head and tail, respectively.
\end{ex}

\end{document}

