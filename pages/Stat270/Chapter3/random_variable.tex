\documentclass{article}

\usepackage{amsmath}

\usepackage{amssymb}

\usepackage{amsthm}

\usepackage{graphicx}



\newtheorem{definition}{Definition}

\newtheorem{property}{Property}

\newtheorem{ex}{Example}

\newtheorem{thm}{Theorem}

\newtheorem{lemma}{Lemma}

\newtheorem{prop}{Proposition}



\begin{document}
\begin{definition}
	A random variable $X$ is a function from a sample space $S$ to a set of numbers $E$ (i.e. $X:S\to E$). If $E=\mathbb{Z}$, then $X$ is said to be a discrete random variable; If $E=\mathbb{R}$, then $X$ is said to be a continuous random variable.
\end{definition}

It might seem so abstract in this way, but what it says is just: X maps the outcomes to some numbers. We will see that in the following examples.
\begin{ex}
	Let there be a fair coin, when I toss it, the sample space $S=\{H,T\}$. Let $X$ be a function defined on $S$ such that
	\[ X(A) = \left\{  \begin{matrix}
	&0 &,if\;A=T\\  &1 &,if\;A=H 
	\end{matrix} \right. \]
	Then $X$ is a discrete random variable. 
\end{ex} 

\begin{ex}
	Suppose I rolled a dice 5 times. Let $X \equiv$ sum of 5 upward-faced numbers I got. This $X$ is a discrete random variable
\end{ex}

\begin{ex}
	Let $Y\equiv$ waiting time at bus stop. This $Y$ is a continuous random variable.
\end{ex}

Since we have defined the random variable, we can now introduce the so-called \textbf{probability mass function}. 
\begin{definition}
	The probability mass function(pmf) of a discrete random variable $X$ is
	\[p_X(x) = P(\{ A\in S| X(A) = x \}) \] 
\end{definition}

The pmf has the following two properties:
\begin{property}
$p_X(x)\geq 0, \forall x\in X(S)$
\end{property}
\begin{property}
	$\sum_x p_X(x) =1$ 
\end{property}


Finally, we can define \textbf{cumulative distribution function} now.
\begin{definition}
	The cumulative distribution function(cdf) is defined as 
	\[ F_X(x) = P(X\leq x) = \sum_{y<x} p_X(y) \]
\end{definition}

\end{document}

