\documentclass{article}

\usepackage{amsmath}

\usepackage{amssymb}

\usepackage{amsthm}

\usepackage{graphicx}



\newtheorem{definition}{Definition}

\newtheorem{property}{Property}

\newtheorem{ex}{Example}

\newtheorem{thm}{Theorem}

\newtheorem{lemma}{Lemma}

\newtheorem{prop}{Proposition}



\begin{document}
\begin{definition}
	The general rule of computing expectation (or easier word, mean) of some real values $g(x)$, is defined by
	\[ E(g(X)) = \sum_x g(x)p_X(x) \] 
\end{definition}

Expectation has a very nice property, it is a linear operator.
\begin{property}
	\[ E\left( \sum_{i=1}^n a_ig_i(X) \right) = \sum_{i=1}^n a_i E(g_i(x))\]
	Where $a_i,g_i(x)\in\mathbb{R}, \forall i=1,2,...,n$.
\end{property}
\begin{proof}
	\begin{align*}
	E\left( \sum_{i=1}^n a_ig_i(X) \right) &= \sum_x \left( \sum_{i=1}^n a_ig_i(X) \right)p_X(x)\\
	&= \sum_x (a_1g_1(x)+a_2g_2(x)+\cdots+a_ng_n(x) )p_X(x)\\
	&= a_1\sum_xg_1(x)p_X(x) + a_2\sum_xg_2(x)p_X(x) + \cdots + a_n\sum_xg_n(x)p_X(x)\\
	&= \sum_{i=1}^n a_iE(g(X))
	\end{align*}
\end{proof}


\begin{definition}
	The expectation value of $X$ is defined by 
	\[ E(X) = \sum_x xp_X(x) \]
\end{definition}

Since the expectation value is linear, we easily see that
\begin{property}
	\[ E(aX+b) = aE(X)+b,\;\;a,b\in\mathbb{R} \]
\end{property}
This property states that if a random variable $X$ is transformed linearly to $aX+b$, then the expectation value of this new random variable is just the transformed version of the original one, namely, $aE(X)+b$.

Geometrically, if all data on the histogram is transformed to some where else, then the measure of location(in this case, mean) will be also transformed in the same manner. \\



\begin{definition}
	The variance of the $X$ is defined by
	\begin{align*}
	Var(X) &= E\left((X-E(X))^2 \right)\\ 
	&= E(X^2) - (E(X))^2\\ 
	&= \sum_x x^2p_X(x) - \left(\sum_x xp_X(x) \right)^2
	\end{align*}
\end{definition}

Variance also has a meaningful property due to the linearity of expectation.
\begin{property}
	\[ Var(aX+b) = a^2Var(X),\;\;a,b\in\mathbb{R} \]
\end{property}
\begin{proof}
	\begin{align*}
	Var(aX+b) &=  E((aX+b)^2) - (E(aX+b))^2\\
	&= E((aX)^2) + E(2abX) + E(b^2) - (aE(X)+b)^2\\
	&= a^2E(X^2) + 2abE(X) + b^2 - a^2(E(X))^2 - 2abE(X) -b^2\\
	&= a^2(E(X^2) - (E(X))^2)\\
	&= a^2Var(X)
	\end{align*}	
\end{proof}
This property states that if data is being shifted left or right, this spread of the data would not changed, thus the value $b$ does not have effect on variance; however, if the scale of the data is shrink or stretched, then the spread of the data would also be changed, thus the value $a$ has effect on variance. 




\end{document}

