\documentclass{article}

\usepackage{amsmath}

\usepackage{amssymb}

\usepackage{amsthm}

\usepackage{graphicx}



\newtheorem{definition}{Definition}

\newtheorem{property}{Property}

\newtheorem{ex}{Example}

\newtheorem{thm}{Theorem}

\newtheorem{lemma}{Lemma}

\newtheorem{prop}{Proposition}



\begin{document}
Binomial distribution is one of most common seen distributions, it is powerful for many situations.

\begin{definition}
	Binomial distribution is a family of distributions, it has two parameters, namely $n$ and $p$. If a discrete random variable $X$ has distribution
	\[ p_X(x) = \left\{ \begin{matrix}
	&\binom{n}{x} p^x(1-p)^{n-x} &,x=0,1,2,...,n\\
	&0 &,\text{otherwise}
	\end{matrix}     \right. \]
	
	we say that $X$ is binomially distributed, i.e. $X\sim Binomial(n,p)$.
\end{definition} 
Of course, the pmf of binomial distribution has the following two properties.
\begin{property}
	$ p(x) \geq 0,\;\;\forall x $
\end{property}
\begin{property}
	$ \sum_x p(x) = 1 $
\end{property}
\begin{proof}
	The first property is obvious. We are going to prove the second property. 
	\[\sum_{x=0}^{n}\binom{n}{x} p^x(1-p)^{n-x} = (1+ (1-p))^n = 1 \] 
	The first equality holds because of the binomial theorem.
\end{proof}

\begin{definition}
	If $X\sim Binomial(n,p)$, then the cumulative distribution function(cdf) of $X$ is defined by
	\[ F_X(x) = P(X\leq x) = \sum_{k=0}^x \binom{n}{k} p^k(1-p)^{n-k} \]
\end{definition}


We should talk about the expectations now.
\begin{prop}
	\begin{align}
	&E(X) = \sum_x x\binom{n}{x} p^x(1-p)^{n-x} = np\\
	&E(X^2) = \sum_x x^2\binom{n}{x} p^x(1-p)^{n-x} = np((n-1)p + 1)\\
	&Var(X) = E(X^2) -(E(X))^2  = np(1-p) 
	\end{align}
\end{prop}
\begin{proof}
	We first prove (1), 
	\begin{align*}
	 \sum_{x=0}^n x\binom{n}{x} p^x(1-p)^{n-x} 
	 &= 0 + \sum_{x=1}^n x\binom{n}{x} p^x(1-p)^{n-x}   \\
	 &= \sum_{x=1}^n \frac{n!}{(x-1)!(n-x)!} p^x(1-p)^{n-x} \\
	 &= np \sum_{x=1}^n \frac{(n-1)!}{(x-1)!(n-x)!} p^{x-1}(1-p)^{n-x}
	\end{align*}
	We now make two substitutions, $x-1\to k$, and $n-1\to m$, then
	\[ E(X) = np \sum_{k=0}^n \frac{m!}{k!(m-k)!} p^k(1-p)^{m-k} = np \]
	
	The equation (2) can be derived in the similar manner, but with a little trick. Notice that $x^2 = x(x-1) + x$, then
	\begin{align*}
		E(X^2) &= E(X(X-1)) + E(X)\\
		&= \sum_{x=0}^n x(x-1)\binom{n}{x} p^x(1-p)^{n-x} + np \\
		&= \sum_{x=2}^n \frac{n!}{(x-2)!(n-x)!} p^x(1-p)^{n-x} + np \\
		&= n(n-1)p^2\sum_{x=2}^n \frac{(n-1)!}{(x-2)!(n-x)!} p^{x-2}(1-p)^{n-x} + np\\
		&= n(n-1)p^2\sum_{k=0}^m \frac{m!}{k!(m-k)!}p^k(1-p)^{n-x} + np\\
		&= n(n-1)p^2 + np\\
		&= np((n-1)p + 1)
	\end{align*} 
	
	Equation (3) follows from the previous two.
	\[ Var(X) = E(X^2) - (E(X))^2 = np((n-1)p+1) -(np)^2 = np(1-p) \]
\end{proof}

\end{document}

