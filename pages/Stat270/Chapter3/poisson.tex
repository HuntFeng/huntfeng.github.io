\documentclass{article}

\usepackage{amsmath}

\usepackage{amssymb}

\usepackage{amsthm}

\usepackage{graphicx}



\newtheorem{definition}{Definition}

\newtheorem{property}{Property}

\newtheorem{ex}{Example}

\newtheorem{thm}{Theorem}

\newtheorem{lemma}{Lemma}

\newtheorem{prop}{Proposition}



\begin{document}
\begin{definition}
	If a discrete random variable $X$ is distributed according Poisson distribution, then it has pmf
	\[ p_X(x) = \left\{  \begin{matrix}
	&\lambda^x \frac{e^{-\lambda}}{x!} &,x=1,2,...\\ 
	&0 &,\text{otherwise}
	\end{matrix} \right. \]
	where $\lambda\in\mathbb{R}^+$ is the parameter of Poisson distribution, and we denote $X\sim Poisson(\lambda)$. Again Poisson distribution is not just one distribution, but a family of a set of distributions with parameter $\lambda$. 
\end{definition}

Poisson distribution is especially good for modeling rare events.

Poisson distribution also has that two properties.
\begin{property}
	$p(x) \geq 0,\;\;\forall x$
\end{property}
\begin{property}
	$\sum_x p(x) = 1 $
\end{property} 
\begin{proof}
	First one is obvious, I am going prove the second one.
	\[\sum_{x=0}^\infty \lambda^x \frac{e^{-\lambda}}{x!} = e^{-\lambda} \sum_{x=0}^\infty \frac{\lambda^x}{x!} = e^{-\lambda}e^{\lambda} =1 \]
\end{proof}


\end{document}

