\documentclass{article}

\usepackage{amsmath}

\usepackage{amssymb}

\usepackage{amsthm}

\usepackage{graphicx}



\newtheorem{definition}{Definition}

\newtheorem{property}{Property}

\newtheorem{ex}{Example}

\newtheorem{thm}{Theorem}

\newtheorem{lemma}{Lemma}

\newtheorem{prop}{Proposition}



\begin{document}
There are three types of discontinuities.

\section{Removable Discontinuity}
\begin{definition}
	A function $f(x)$ has a removable discontinuity at $x=x_0$ if
	\[ \lim_{x\to x_0^-} f(x)  = \lim_{x\to x_0^+} f(x) \neq f(x_0) \]
\end{definition}

\begin{ex}
	the rational function
	\[ f(x) = \frac{x^2+2x+1}{x+1} \]
	has a removable discontinuity at $x=-1$.
	
	Since at $x=-1$ this is a hole. So the left right limits exist and equal, but the function value is missing.
	
	\begin{center}
		!!!!!!!!!!!!!!!!!!!!!! desmos graph !!!!!!!!!!!!!!!!!!!!!!!!!!
	\end{center}
\end{ex}


\section{Jump Discontinuity}
\begin{definition}
	A function is said to have a jump discontinuity at $x=x_0$ if 
	\[ \lim_{x\to x_0^-} f(x) \neq \lim_{x\to x_0^+} f(x) \]
\end{definition}


\begin{ex}
	The piecewise function
	\[ f(x) = \left\{ \begin{matrix}
	&x,&x<0 \\ &\cos x, &x\geq 0
	\end{matrix} \right. \] 
	has a jump discontinuity at $x=0$. Since $\lim_{x\to 0^-}f(x) = \lim_{x\to 0^-} x = 0$, yet $\lim_{x\to 0^+} f(x) = \lim_{x\to 0^+} \cos x = 1$.
	
	
	\begin{center}
		!!!!!!!!!!!!!!!!!!!!!! desmos graph !!!!!!!!!!!!!!!!!!!!!!!!!!
	\end{center}
\end{ex}

\section{Essential Discontinuity}
\begin{definition}
	A function has essential discontinuity at $x=x_0$ if one (both) of one-sided limits does(do) not exist or be infinity.  
\end{definition}

\begin{ex}
	\[ f(x) = \sin\left(\frac{1}{x} \right) \]
	
	has an essential discontinuity at $x=0$, since both left and right limits at $x=0$ do not exist.
	
	\begin{center}
		!!!!!!!!!!!!!!!!!!!!!! desmos graph !!!!!!!!!!!!!!!!!!!!!!!!!!
	\end{center}
\end{ex}

\end{document}

