\documentclass{article}

\usepackage{amsmath}

\usepackage{amssymb}

\usepackage{amsthm}

\usepackage{graphicx}



\newtheorem{definition}{Definition}

\newtheorem{property}{Property}

\newtheorem{ex}{Example}

\newtheorem{thm}{Theorem}

\newtheorem{cor}{Corollary}

\newtheorem{lemma}{Lemma}

\newtheorem{prop}{Proposition}



\begin{document}
\begin{definition}
	A function $f(x)$ is continuous at point $x=x_0$ if and only if the following conditions are satisfied: 
	\begin{align}
	&\lim_{x\to x_0^-} f(x),\; \lim_{x\to x_0^+} f(x),\;\text{and}\; f(x_0)\; \text{exist}.\\
	&\lim_{x\to x_0^-} f(x)=\lim_{x\to x_0^+} f(x)=f(x_0)
	\end{align}
\end{definition}

\begin{ex}
	Show that $f(x)$ is continuous at $x=1$ if it is defined by
	\[ f(x) = \left\{ \begin{matrix}
	&x+1,\;&x<1 \\ &x^2+x,&x\geq 1 
	\end{matrix} \right. \]
	
	\begin{proof}
		We have to know what left/right limit and also the function value at $x=2$ are  
		\begin{align*}
		&\lim_{x\to 1^-} f(x) = \lim_{x\to 1^-} x+1 = 2\\
		&\lim_{x\to 1^+} f(x) = \lim_{x\to 1^+} x^2+x = 2\\
		&f(1) = 1^2 + 1 = 2\\
		\end{align*}
		 
		We see that, they all exist, so condition (1) is satisfied. Moreover, since they are all equal, so condition (2) is also satisfied. Thus $f(x)$ is continuous at $x=1$.
	\end{proof}
\end{ex}


\begin{ex}
	Find a number $k$ such that the function 
	\[ f(x) = \left\{ \begin{matrix}
	&\sin(x+k), &x<0\\ &e^x, &x\geq 0
	\end{matrix}  \right. \] 
	is continuous at $x=0$.
	
	
	\begin{proof}
		In order to make this function continuous at $x=1$, we have to make it satisfies the two conditions.
		
		Since 
		\begin{align*}
		&\lim_{x\to 0^-} f(x) = \sin k\\
		&\lim_{x\to 0^+} f(x) = e^0 = 1\\
		&f(0) = e^0 = 1
		\end{align*} 
		
		we see that we have to set $\sin k=1$. Thus $k=\frac{\pi}{2} + 2\pi n,\;n\in\mathbb{Z}$. 
	\end{proof}
\end{ex}

\begin{thm}
	\textbf{Intermediate Value Theorem}.
	
	If a function $f(x)$ is continuous on [a,b], then for any $\xi\in f([a,b])$ , we can find a $a\leq c\leq b$ such that 
	\[ f(x) = \xi  \]
\end{thm}

\begin{ex}
	Since $f(x)=x^2$ is continuous on $[0,2]$, and $f([0,2])=[0,4]$. Since $\pi\in[0,4]$, thus we can find a $c\in[0,2]$ such that $f(c) = \pi$.
\end{ex}

\begin{cor}
	If a function $f(x)$ is continuous on [a,b], and $f(a)f(b)<0$, then there must be at least one $c\in[a,b]$ such that $f(c)=0$.
\end{cor}
\begin{proof}
	Since $f(a)f(b)<0$, it means one and only one of $f(a)$ and $f(b)$ is negative. Without loss of generality, we assume $f(a)<0<f(b)$. Then since $f(x)$ is continuous on [a,b], thus by intermediate value theorem, there must be at least one $c$ such that $f(c)=0$.
\end{proof}

\begin{ex}
	Show that $x^20 + 3x^19 -1 = 0$ must have at least one solution in [0,1].
	\begin{proof}
		Since the function $f(x) = x^20 + 3x^19 -1$ is continuous on [0,1], and $f(0)f(1)=(-1)\times(3) <0$. Thus it must have one zero in [0,1].
	\end{proof}
\end{ex}


\end{document}

