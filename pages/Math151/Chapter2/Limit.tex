\documentclass{article}

\usepackage{amsmath}

\usepackage{amssymb}

\usepackage{amsthm}

\usepackage{graphicx}



\newtheorem{definition}{Definition}

\newtheorem{property}{Property}

\newtheorem{ex}{Example}

\newtheorem{thm}{Theorem}

\newtheorem{lemma}{Lemma}

\newtheorem{prop}{Proposition}



\begin{document}
\begin{definition}
	 We say that a function's limit exits at $x= x_0$,i.e 
	 \[ \lim_{x\to x_0} f(x) = L \]
	 if this statement is true: $\forall \epsilon>0, \exists \delta>0$ such that if $0<|x-x_0|<\delta$, then
	\[ |f(x)-L| < \epsilon \]
\end{definition}

\begin{ex}
	Let $f(x) = \frac{(x+1)(x-1)}{(x+1)x} $.
	\begin{center}
	!!!!!!!!!!!!!!!! demos graph !!!!!!!!!!!!!!!!!!!!!!
	\end{center}
We see that from the graph 
\[ \lim_{x\to1} f(x) = 0 \]
\[ \lim_{x\to0} f(x) = DNE\]
\[ \lim_{x\to-1} f(x) = 2 \]

\end{ex}

Notes: The limit of a function at some points $x_0$ has nothing to do with the function value at that point.

\begin{thm}
	\begin{align*}
	&\lim (f(x) \pm g(x) = \lim f(x) \pm \lim g(x)\\
	&\lim (f(x) \times g(x) = \lim f(x) \times \lim g(x)\\
	&\lim \frac{f(x)}{g(x)} =\frac{ \lim f(x)}{\lim g(x)}\\
	\end{align*}
\end{thm}

\begin{ex}
	\[\lim_{x\to 0}(\cos x+ \sin x) = \lim_{x\to 0}\cos x + \lim_{x\to 0}\sin x = 1+ 0 =1 \]
\end{ex}

\begin{ex}
	\[ \lim_{x\to 1} \frac{x^2}{x+1} = \frac{\lim_{x\to1} x^2}{\lim_{x\to 1}(x+1)} = \frac{1}{2} \]
\end{ex}
 



\end{document}

