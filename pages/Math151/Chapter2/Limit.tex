\documentclass{article}

\usepackage{amsmath}

\usepackage{amssymb}

\usepackage{amsthm}

\usepackage{graphicx}



\newtheorem{definition}{Definition}

\newtheorem{property}{Property}

\newtheorem{ex}{Example}

\newtheorem{thm}{Theorem}

\newtheorem{lemma}{Lemma}

\newtheorem{prop}{Proposition}



\begin{document}
\section{Introduction}
\begin{definition}
	 We say that a function's limit exits at $x= x_0$,i.e 
	 \[ \lim_{x\to x_0} f(x) = L \]
	 if this statement is true: $\forall \epsilon>0, \exists \delta>0$ such that if $0<|x-x_0|<\delta$, then
	\[ |f(x)-L| < \epsilon \]
\end{definition}

\begin{ex}
	Let $f(x) = \frac{(x+1)(x-1)}{(x+1)x} $.
	\begin{center}
	 !!!!!!!!!!!!demos !!!!!!!!!!!!!!!!!!!!!!!!!!!!!!!!!
	\end{center}
We see that from the graph 
\[ \lim_{x\to1} f(x) = 0 \]
\[ \lim_{x\to0} f(x) = DNE\]
\[ \lim_{x\to-1} f(x) = 2 \]

\end{ex}

Notes: The limit of a function at some points $x_0$ has nothing to do with the function value at that point.

\begin{thm}
	\begin{align*}
	&\lim (f(x) \pm g(x) = \lim f(x) \pm \lim g(x)\\
	&\lim (f(x) \times g(x) = \lim f(x) \times \lim g(x)\\
	&\lim \frac{f(x)}{g(x)} =\frac{ \lim f(x)}{\lim g(x)}\\
	\end{align*}
\end{thm}

\begin{ex}
	\[\lim_{x\to 0}(\cos x+ \sin x) = \lim_{x\to 0}\cos x + \lim_{x\to 0}\sin x = 1+ 0 =1 \]
\end{ex}

\begin{ex}
	\[ \lim_{x\to 1} \frac{x^2}{x+1} = \frac{\lim_{x\to1} x^2}{\lim_{x\to 1}(x+1)} = \frac{1}{2} \]
\end{ex}
 
\begin{ex}
	\[\lim_{x\to-1}\frac{x^2 + 3x + 2}{x+1}\]
	
	If we simply plug in the value $x=-1$, we will have a 0/0 situation, this is not good. However, we see that the the term causing the problem, $x+1$, can be canceled. Thus
	\begin{align*}
		\lim_{x\to-1} \frac{x^2 +3x +2 }{x+1} 
		&= \lim_{x\to-1} \frac{(x+1)(x+2)}{x+1}\\
		&= \lim_{x\to-1} x+2\\
		&= 1
	\end{align*}
	
	Although we know that the function value at $x=-1$ is still undefined, but its limit at $x=-1$ has nothing to do with the function.   
\end{ex}

\begin{ex}
	\[\lim_{x\to2}\frac{x-2}{\sqrt{x-1} -1}\]
	
	If we simply put the the value $x=2$, we again encounter the 0/0 situation.
	However, when we see square root, try to use the square difference formula to simplify the expression first.
	
	\begin{align*}
	\lim_{x\to2}\frac{x-2}{\sqrt{x-1} -1} 
	&= \lim_{x\to2}\frac{(x-2)(\sqrt{x-1}+1)}{(\sqrt{x-1} -1)(\sqrt{x-1}+1)}\\
	&= \lim_{x\to2} \frac{(x-2)(\sqrt{x-1}+1)}{x-2}\\
	&= \lim_{x\to2} \sqrt{x-1}+1\\
	&= 2 
	\end{align*} 
\end{ex}


Note: When dealing with limit, if some situations occur(e.g 0/0), remember to simplify or adjust the expression before you further calculate.

\section{One-sided limit}
\begin{definition}
	The \textbf{left(right) limit} of a function, $f(x)$, at $x=x_0$ is the value the function is approaching when x is approaching $x_0$ from the left(right).  The notations for them are
	\begin{align}
	&\lim_{x\to x_0^-}f(x), &\text{left limit}\\
	&\lim_{x\to x_0^+}f(x), &\text{right limit}
	\end{align}
\end{definition}

\begin{ex}
	Let $f(x)= \frac{1}{x}$. Then the one-sided limits at $x=0$ are
	\begin{align*}
	&\lim_{x\to 0^-}f(x) = -\infty, &\text{left limit}\\
	&\lim_{x\to 0^+}f(x)= -\infty, &\text{right limit}
	\end{align*}
\end{ex}


\begin{thm}
	$\lim_{x\to x_0} f(x)$ exists if and only if the left and right limit exist and equal, i.e. $\lim_{x\to x_0^-}f(x)=\lim_{x\to x_0^+}f(x)$.
\end{thm}

\begin{ex}
	\[f(x) = \frac{(x+1)}{(x+1)x}\]
	The limit of this function at $x=-1$ exists since both left and right limit exist and equal.
	
	\begin{center}
		!!!!!!!!!!!!!!!!!!!!!!!desmos!!!!!!!!!!!!!!!!!!!!!!!!!
	\end{center}
\end{ex}


\end{document}

