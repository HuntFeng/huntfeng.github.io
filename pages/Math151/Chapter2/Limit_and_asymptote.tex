\documentclass{article}

\usepackage{amsmath}

\usepackage{amssymb}

\usepackage{amsthm}

\usepackage{graphicx}



\newtheorem{definition}{Definition}

\newtheorem{property}{Property}

\newtheorem{ex}{Example}

\newtheorem{thm}{Theorem}

\newtheorem{lemma}{Lemma}

\newtheorem{prop}{Proposition}



\begin{document}
\begin{definition}
	If $\forall \epsilon >0$, $\exists N>0$ such that when $x>N$, then 
	\[ |f(x) -L |< \epsilon \]
	where $L\in\mathbb{R}$. We say that the limit of the function when $x\to \infty$ is $L$. In other words,
	\[ \lim_{x\to\infty} f(x) = L \]
\end{definition}

\begin{ex}
	One very important example that worths to remember is this
	\[ \lim_{x\to\infty} \frac{1}{x^n} =0\;\;(n\in\mathbb{N}) \]
	We can easily see this from the graph.
	\begin{center}
		!!!!!!!!!!!!!!!!!!!! demos graph !!!!!!!!!!!!!!!!!!!!!!!!!
	\end{center}
\end{ex}

\begin{ex}
	\[ lim_{x\to\infty} = \frac{3x^2+2x+1}{2x^2-3} \]
	Graphically, we know that this just asking us to find the horizontal asymptote of this rational function. And the answer is 3/2.
	
	Algebraically, for the numerator and the denominator, we both divide the leading term of the polynomial with smaller degree. In this case, we divide $x^2$.
	\begin{align*}
	lim_{x\to\infty} \frac{3x^2+2x+1}{2x^2-3}
	&= \lim_{x\to\infty} \frac{(3x^2+2x+1)/x^2}{(2x^2-3)/x^2}\\
	&= \lim_{x\to\infty} \frac{3 + 2\frac{1}{x}+ \frac{1}{x}}{2-3\frac{1}{x}}\\
	&= \frac{3}{2}
	\end{align*}
\end{ex}


\begin{ex}
	\[ \lim_{x\to\infty} \frac{\sqrt{x^2+2}}{x} \]
	When $x$ is very large, we see that $x^2+2 \approx 2$. Then
	\[\lim_{x\to\infty} \frac{\sqrt{x^2+2}}{x} = \lim_{x\to\infty} \frac{\sqrt{x^2}}{x} \]
	
	One important thing to remember is that $\sqrt{x^2}=|x|$. So

	\[\lim_{x\to\infty} \frac{\sqrt{x^2}}{x}  = \lim_{x\to\infty} \frac{|x|}{x}
	= 1\]
\end{ex}


\begin{ex}
	\[ \lim_{x\to\infty} \sqrt{x^2-x} + x \]
	Be careful when dealing with square root.
	
	\begin{align*}
	\lim_{x\to\infty} \sqrt{x^2-x} + x &= \lim_{x\to\infty} \frac{\sqrt{x^2-x} + x}{1}\\
	&= \frac{(x^2 -x) - x^2}{\sqrt{x^2-x} - x}\\
	&= \frac{-x}{\sqrt{x^2-x} - x}\\
	&= \frac{-1}{\frac{\sqrt{x^2-x}}{x} - 1}\\
	&= \frac{-1}{\frac{|x|}{x}-1}\\
	&= \frac{-1}{-1-1}\\
	&= 2
	\end{align*}
\end{ex}




\end{document}

