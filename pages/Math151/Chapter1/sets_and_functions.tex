\documentclass{article}

\usepackage{amsmath}

\usepackage{amssymb}

\usepackage{amsthm}

\usepackage{graphicx}



\newtheorem{definition}{Definition}

\newtheorem{property}{Property}

\newtheorem{ex}{Example}

\newtheorem{thm}{Theorem}

\newtheorem{lemma}{Lemma}

\newtheorem{prop}{Proposition}



\begin{document}
\begin{definition}
	A mapping, $\varphi$, defined on set $D$, is a rule that assign each element in $D$ a value. The set $D$ is called the domain of the mapping, and the set  of values we get (let's give it a name, $R$) after applying the mapping is called the range of the mapping. We denote this by
	\[ \varphi: D\to R \]
\end{definition}

\begin{definition}
	A function $f:D\to R$ is a special mapping in which $\forall x\in D$, there is only one $y\in R$ such that $y = f(x)$.
\end{definition}
\begin{ex}
	\begin{align*}
	&y=3x-1,\;D=\mathbb{R}\;\; &\text{is a function}\\
	&y= x^2+2x-1,\;D=\mathbb{R}\;\; &\text{is a function}\\
	&x^2+y^2 =1,\;D=[-1,1]\;\; &\text{is not a function}\\
	&y^2 = x\;D=\mathbb{R^+}\;\; &\text{is not a function}
	\end{align*}
\end{ex}

\begin{definition}
	A function $f:D\to R$ is said to be 1-1 if $\forall y\in R$, there is only one $x\in D$ such that $y = f(x)$. 
\end{definition}

\begin{ex}
	\begin{align*}
	&y=3x-1,\;D=\mathbb{R}\;\; &\text{is a 1-1 function}\\
	&y= x^2+2x-1,\;D=\mathbb{R}\;\; &\text{is not a 1-1 function}\\
	&y= x^2+2x-1,\;D=\mathbb{R^+}\;\; &\text{is a 1-1 function}\\
	\end{align*}
\end{ex}

\begin{definition}
	An inverse function of a function $f:D\to R$ is a mapping from $R$ to $D$, and we denote it as $f^{-1}:R\to D$. Its action is defined by 
	\[ f^{-1}(f(x)) = x \]
\end{definition}
\begin{ex}
	Let $f(x)= 3x+1$, then its inverse function $f^{-1}$ can be calculated as follow:
	\begin{align*}
	&y=3x+1\;\;&\text{change the $f(x)$ symbol to y}\\
	&x = \frac{y-1}{3}\;\;&\text{express $x$ in terms of y}\\
	&y = \frac{x-1}{3}\;\; &\text{switch the symbol of x and y}\\
	&f^{-1}(x) = \frac{x-1}{3}\;\;&\text{change the symbol $y$ to $f^{-1}(x)$} 
	\end{align*}
	
	What inverse function does is the inverse action of the original function. For example: 
	\begin{align*}
	&f(1) = 4\;\;\text{(f maps 1 to 4)} &f^{-1}(4) = 1\;\;\text{($f^{-1}$ maps 4 back to 1)}\\
	&f(2) = 7\;\;\text{(f maps 2 to 7)} &f^{-1}(7) = 2\;\;\text{($f^{-1}$ maps 7 back to 2)}\\
	\end{align*}
\end{ex}
\begin{prop}
	The graph of the function and its inverse are symmetric about the $y=x$ line. 
\end{prop}
\begin{proof}
	It is not hard to see that if the graph of the function consists of points of the form $(x,y)$, where $x\in D$ and $y\in R$, then the inverse function takes the values in $R$ and maps them back to $D$, thus the graph of the inverse function is consisted of points of the form $(y,x)$.
\end{proof}
\begin{ex}
	The following is the graph of $f(x)=3x+1$ and its inverse.
	\begin{center}
		\includegraphics*[width=12cm]{inverse_graph.png}
	\end{center}
\end{ex}




\end{document}

