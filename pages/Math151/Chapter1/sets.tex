\documentclass{article}

\usepackage{amsmath}

\usepackage{amssymb}

\usepackage{amsthm}

\usepackage{graphicx}



\newtheorem{definition}{Definition}

\newtheorem{property}{Property}

\newtheorem{ex}{Example}

\newtheorem{thm}{Theorem}

\newtheorem{lemma}{Lemma}

\newtheorem{prop}{Proposition}



\begin{document}
\begin{definition}
	A set is just a collection of objects satisfying certain properties. If an object(say $x$) is belong to a set(say $A$), then we say $x$ is an element of $A$, and denote it as $x\in A$. 
\end{definition}

\begin{definition}
	A set $A$ is an subset of $B$ if $\forall x\in A$, then $x\in B$. And we denote this by $A\subseteq B$. Moreover, if $A$ is an subset of $B$ and there $A$ does not contain all elements of $B$, we call $A$ as a proper subset of $B$, and we denote this by $A\subset B$. 
\end{definition}

\begin{ex}
	Let $A=\{ 0,1,2,{3,4,5} \}$. Then 
	\begin{align*}
	&2\in A\\
	&{3,4,5}\in A\\
	&{0,1} \subset A\\
	&{0,{3,4,5}} \subset A\\
	\end{align*}
\end{ex}


\subsection*{Important sets}
\begin{ex}
	\begin{align*}
	&\mathbb{N}\; \text{is the set of all natural numbers.}\\
	&\mathbb{Z}\; \text{is the set of all integers.}\\
	&\mathbb{Q}\; \text{is the set of all real numbers.}\\
	&\mathbb{R}\; \text{is the set of all rational numbers.}\\
	&\mathbb{R}\backslash \mathbb{Q}\; \text{is the set of all irrational numbers.}
	\end{align*}
\end{ex}

\subsection*{Curly Bracket notation}
\begin{ex}
	$ \{x|x>4\} $ is the set of real number $x$, such that $x$ is greater than 4.
\end{ex}

\begin{ex}
	$\{ x\in\mathbb{Q}|x<-1\}$ is the set of rational numbers $x$ such that $x$ is less that -1.
\end{ex}

\subsection*{Interval notation}
\begin{ex}
	$[1,2]$ is the set of all real numbers between 1 and 2, including the end points 1 and 2.\\
	$(-1,2)$ is the set of all real numbers between -1 and 2, excluding the end points 1 and 2.\\
	$(-1,3]$ is the set of all real numbers between -1 and 3, endpoint 3 is included but -1 is excluded.\\
	$[-2,3)$ is the set of all real numbers between -2 and 3, -2 is included but 3 is excluded.\\
	$[0,\infty)$ is the set of all real numbers from 0 to $\infty$, 0 is included.\\
	$(-\infty, 1]$ is the set of all real numbers from $-\infty$ to 1, 1 is included.
\end{ex}



\subsection*{Set Operations}
There are 3 basic operations we have to learn.
\begin{definition}
	Union of two sets $A$ and $B$, $A\cup B$ is defined by joining all elements in set $A$ and $B$ together.
\end{definition}
\begin{ex}
	Let $A= [3,5)$, and $B= [5,\infty)$, then $A\cup B = [3,\infty)$ 
\end{ex}

\begin{definition}
	Intersection of two sets $A$ and $B$, $A\cap B$ is defined by taking the intersection(common part) of two sets.
\end{definition}
\begin{ex}
	Let $A= [3,5]$ and $B=(-2,4)$, then $A\cap B = [3,4)$.
\end{ex}

\begin{definition}
	Difference between two sets $A$ and $B$, usually denoted by $A\backslash B$ or $A-B$, is defined by taking all elements of $A$ except those are also belong to $B$.
\end{definition}
\begin{ex}
	The set of all irrational numbers consists of all real numbers that are not rational. In other words, $\mathbb{R}\backslash\mathbb{Q}$ (or maybe you can denote it by $\mathbb{R}-\mathbb{Q}$). 
\end{ex}




\end{document}

