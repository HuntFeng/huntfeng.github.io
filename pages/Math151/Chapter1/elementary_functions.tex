\documentclass{article}

\usepackage{amsmath}

\usepackage{amssymb}

\usepackage{amsthm}

\usepackage{graphicx}



\newtheorem{definition}{Definition}

\newtheorem{property}{Property}

\newtheorem{ex}{Example}

\newtheorem{thm}{Theorem}

\newtheorem{lemma}{Lemma}

\newtheorem{prop}{Proposition}



\begin{document}
In grade 12, we learned a few elementary functions. 
\subsection*{Polynomials}
A polynomial is a function having the form of
\[a_nx^n + a_{n-1}x^{n-1}+\cdots+a_1x +a_0\]
where the powers of $x$ must be integer and $a_i\in\mathbb{R}$.
  
\begin{property}
	A odd degree(n is odd) polynomial goes through quadrant I,III if the leading coefficient of it is positive($a_n>0$), and goes through quadrant II,IV when leading coefficient is negative($a_n<0$).
	
	While a even degree(n is even) polynomial goes through quadrant I,II if the leading coefficient of it is positive($a_n>0$), and goes through quadrant III,IV when leading coefficient is negative($a_n<0$).
\end{property}

\begin{center}
	!!!!!!!!!!!!!!!! demos graph !!!!!!!!!!!!!!!!!!!!!!!
\end{center}
\subsection*{Rational functions}
Rational functions are functions having the following form 
\[ \frac{p(x)}{q(x)} = \frac{a_nx^n + a_{n-1}x^{n-1}+\cdots+a_1x +a_0 }{b_mx^m + b_{m-1}x^{m-1}+\cdots+b_1x +b_0} \]
where we can see that $p(x)$ and $q(x)$ are both polynomials.

\begin{property}
	Vertical Asymptote: It occurs at those $x$'s which make the denominator 0.
	
	Hole: It occurs at those $x$'s which make the denominator 0 and the term that causing this can be canceled in the function.
	
	Horizontal Asymptote: When $n<m$, the horizontal asymptote will occur at $y=0$;
	when $n=m$, the horizontal asymptote will occur at $y=a_n/b_n$.
\end{property}

\subsection*{Exponential Functions}
\begin{center}
	!!!!!!!!!!!!!!!! demos graph !!!!!!!!!!!!!!!!!!!!!!!
\end{center}
\subsection*{Logrithms}
Logarithmic functions are the inverse of the exponential functions, thus logarithmic functions and exponential functions are symmetric about the $y=x$ line.
\begin{center}
	!!!!!!!!!!!!!!!! demos graph !!!!!!!!!!!!!!!!!!!!!!!
\end{center}

\subsection*{Trigonometric Functions}
\begin{center}
	!!!!!!!!!!!!!!!! demos graph !!!!!!!!!!!!!!!!!!!!!!!
\end{center}

\end{document}

