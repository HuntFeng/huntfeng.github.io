\documentclass{article}

\usepackage{amsmath}

\usepackage{amssymb}

\usepackage{amsthm}

\usepackage{graphicx}



\newtheorem{definition}{Definition}

\newtheorem{property}{Property}

\newtheorem{ex}{Example}

\newtheorem{thm}{Theorem}

\newtheorem{lemma}{Lemma}

\newtheorem{prop}{Proposition}



\begin{document}
There are few important transformations we have to know. We first start by talking about linear transformation of a function. 
\subsection*{Linear Transformations}
By apply the linear transformations on the graph, shifts, stretch and compression: 
\[f(x) \to af(b(x+c)) + d \]
the point $(x,y)$ on the original graph will transformed to $(\frac{x}{b}-c,ay+d)$.


There are four linear transformations on a function $f(x)$, controlled by four parameters a,b,c,d. 

We first talk about the actions of c and d.

c controls the horizontal shift and d controls the vertical shift.

\begin{ex}
	!!!!!!!!!!!!!!!!!!!!!!!!!!!!! add demos graph here !!!!!!!!!!!!!!!!!!!!!!!!
\end{ex}


a and b controls horizontal and vertical stretch and compression. However, their behaviors are a little different. For a, it stretches(if $|a|>1$) or compresses(if $0<|a|<1$) the graph vertically by a factor factor of $|a|$, and if $a<0$, then the graph will be flipped about the x-axis. Meanwhile, b stretches(if $0<|b|<1$) or compresses(if $|b|>1$) the graph by a factor of $1/|b|$. If $b<0$, the graph is flipped about the y-axis.

\begin{ex}
	!!!!!!!!!!!!!!!!!!!!!!!!!!!!! add demos graph here !!!!!!!!!!!!!!!!!!!!!!!!
\end{ex}
      
\subsection*{Absolute Value}
Let's see what can absolute value function do on the graphs. 

Typically, there are two most common-used such transformations in Calculus.
\begin{align}
f(x) \to |f(x)|\\
f(x) \to f(|x|)
\end{align} 

The first one is easy, the absolute value function flips the part where the function values $f(x)<0$ about the x-axis.
\begin{ex}
	!!!!!!!!!!!!!!!!!!!!!!!!!!!!! add demos graph here !!!!!!!!!!!!!!!!!!!!!!!!
\end{ex}

The second one is a little bit tricky. The absolute value function only acts on the variable $x$, we can see that $f(|x|)$ actually becomes an even function since 
\[ f(|-x|) = f(|x|) \]
and we by the definition of absolute value function,
\[ f(|x|) = \left\{ \begin{matrix}
f(x)\;,x\geq 0\\ f(-x)\;,x<0
\end{matrix}  \right. \] 
we see that after the transformation, the left part($x<0$) of the graph disappeared and became the the mirror image of the right part($x>0$) of the graph.
\begin{ex}
	!!!!!!!!!!!!!!!!!!!!!!!!!!!!! add demos graph here !!!!!!!!!!!!!!!!!!!!!!!!
\end{ex} 




\end{document}

