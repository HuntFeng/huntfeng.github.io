\documentclass{article}

\usepackage{amsmath}

\usepackage{amssymb}

\usepackage{amsthm}

\usepackage{graphicx}



\newtheorem{definition}{Definition}

\newtheorem{property}{Property}

\newtheorem{ex}{Example}

\newtheorem{thm}{Theorem}

\newtheorem{lemma}{Lemma}

\newtheorem{prop}{Proposition}



\begin{document}
\section{Average Rate of Change}
\begin{definition}
	The average rate of change of a continuous function $f(x)$ during the time interval $[a,b]$ is defined by 
	\[ \frac{\Delta y}{\Delta x} = \frac{f(b)-f(a)}{b-a} \]
\end{definition}

\begin{ex}
	Suppose a car is moving with constant velocity, and its position can be described by function $r(t) = 5t$ (t in second, r in meter). Then its average rate of change(in this case, average velocity) of the car during the time interval $[0,2]$ can be calculated by
	\[ \frac{\Delta r}{\Delta t} = \frac{r(2)-r(0)}{2-0} = \frac{10-0}{2-0}=5 \]
	In other words, the average velocity of the car is 5m/s. 
\end{ex}

\begin{ex}
	Suppose a chemical reaction occurs between some materials A and B, and the mass of the material A can be described by a function $m_A(t) = \frac{5}{t+1}$ (t in second, mass in gram). Then the average rate of change(in this case, average reaction rate) during the time interval $[1,2]$ can be computed by 
	 \[ \frac{\Delta m_A}{\Delta t} = \frac{m_A(2)-m_A(1)}{2-1} = \frac{5/3 - 5/2}{1} = -\frac{5}{6} \]
	 The average reaction rate of this reaction is $-\frac{5}{6}$gram/s.
\end{ex}


\section{Geometrical Interpretation}
\begin{definition}
	The slope of a secant line that intersecting the function $f(x)$ at two points $(a,f(a))$ and $(b,f(b))$ can be calculated by
	\[ slope = \frac{\Delta y}{\Delta x} = \frac{f(b)-f(a)}{b-a} \]
	
	\begin{center}
		\includegraphics*[width=7cm]{avroc.png}
	\end{center}
\end{definition}

\begin{ex}
	Draw a secant line that intersects the function $f(x)=x^2$ at $x=1$ and $x=2$, then its slope is
	\[ \frac{\Delta y}{\Delta x} = \frac{f(2)-f(1)}{2-1} = \frac{2^2-1^2}{1} = 3 \]
\end{ex}

The average rate of change of a function $f(x)$ during the time interval $[a,b]$ is the same as the slope of the secant line intersecting the function $f(x)$ at points $x=a$ and $x=b$.



\end{document}

