\documentclass{article}
\usepackage{amsmath}
\usepackage{amssymb}
\usepackage{amsthm}
\usepackage{graphicx}    
    
\newtheorem{definition}{Definition}
\newtheorem{property}{Property} 
\newtheorem{ex}{Example}
\newtheorem{thm}{Theorem}
\newtheorem{lemma}{Lemma}
\newtheorem{prop}{Proposition}  
\newtheorem*{solution}{Solution}
    

\begin{document}
Sometimes, a the dependent variable y cannot easily written explicitly as a function of x. Therefore it is important that we can still find out the derivative of the variable y with respect to x. To do this, we use the chain rule.


\begin{ex}
    Find the slope of the tangent line of the graph
    \[ x^2+y^2 = 1 \]

    \begin{solution}
    We take the derivative of the equation with respect to x for both sides, we have
    \begin{align*}
        \frac{d}{dx} (x^2+y^2) = \frac{d}{dx}0\\
        2x + 2yy' = 0\\
        y' = -\frac{x}{y}
    \end{align*}

    The reason why the derivative of the second term is 2yy' is because we assume that y is a function of x, therefore $y^2$ is a composite function which the "outer function" is square and the "inner function" is $y(x)$.  
    \end{solution}
\end{ex}

\end{document}