\documentclass{article}

\usepackage{amsmath}

\usepackage{amssymb}

\usepackage{amsthm}

\usepackage{graphicx}



\newtheorem{definition}{Definition}

\newtheorem{property}{Property}

\newtheorem{ex}{Example}

\newtheorem{thm}{Theorem}

\newtheorem{lemma}{Lemma}

\newtheorem{prop}{Proposition}

\newtheorem{problem}{Problem}

\newtheorem{solution}{Solution}



\begin{document}
\begin{definition}
	A general form of a plane is 
	\[ Ax + By + Cz = D \]
	It is a set of points that satisfy this equation.
\end{definition}

However, practically the general form is not as useful as the normal-point form.

\begin{definition}
	The \textbf{normal-point} form of a plane is defined by the normal vector $\vec{n} = (n_x,n_y,n_z)$ of the plane and a point $\vec{r}_0 = (x_0,y_0,z_0)$ the plane contains.
	\[ \vec{n}\cdot(\vec{r}-\vec{r}_0) = 0 \]
	or 
	\[ n_x(x-x_0) + n_y(y-y_0) + n_z(z-z_0) = 0 \]
	
	\begin{center}
		\includegraphics*[width=5cm]{plane.png}
	\end{center}
	
\end{definition} 

\begin{problem}
	Determine the equation of the plane which contains the three points $\vec{P}=(1,-2,0),\vec{Q}=(3,1,4),\vec{R}=(0,-1,2)$. 
\end{problem}
\begin{solution}
	First we need to find out the normal of this plane.
	\begin{align*}
	\vec{n} &= \overrightarrow{PQ}\times\overrightarrow{PR}\\ 
	&= (\vec{Q}-\vec{P})\times(\vec{R}-\vec{P})\\
	&= (2,3,4) \times (-1,1,2)\\
	&= (2,-8,5)
	\end{align*}
	
	Since the plane contains all these points, we choose one, say P, then the normal-point form equation of this plane is 
	\[ 2(x-1)-8(y+2)+5(z-0)=0 \]
	or written in general form
	\[ 2x-8y+5z = 18 \] 
\end{solution}


\end{document}

