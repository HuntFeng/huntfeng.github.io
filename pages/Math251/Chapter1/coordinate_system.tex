\documentclass{article}

\usepackage{amsmath}

\usepackage{amssymb}

\usepackage{amsthm}

\usepackage{graphicx}



\newtheorem{definition}{Definition}

\newtheorem{property}{Property}

\newtheorem{ex}{Example}

\newtheorem{thm}{Theorem}

\newtheorem{lemma}{Lemma}

\newtheorem{prop}{Proposition}



\begin{document}

\section{Rectangular coordinate system}
This is trivial, there are three coordinates, they are x, y and z, respectively. A vector in this coordinate with a tip at position (x,y,z) and tail at the origin is usually denoted by $(x,y,z)$ or $\langle x,y,z \rangle$ or 3 by 1 matrix $[x,y,z]^T$.

the distance between two points $(x_1,y_1,z_1)$ and $(x_2,y_2,z_2)$ can be calculated by 
\[  \sqrt{(x_1-x_2)^2+(y_1-y_2)^2+(z_1-z_2)^2} \]


\section{Cylindrical coordinate system}
\begin{center}
	\includegraphics*[width=7cm]{cylindrical-coordinates.png}
\end{center}

In cylindrical coordinate, instead of using x, y and z as coordinates we express the vector in terms of the length of the vector $r$, azimuth angle $\phi$, and height $z$, i.e. $(r,\phi,z)$. The transformation is shown below:  
\begin{align*}
x &= r\cos\phi \\
y &= r\sin\phi\\
z &= z
\end{align*}

or 

\begin{align*}
r &= \sqrt{x^2+y^2},\; r\in[0,\infty)\\
\phi &= \arctan(y/x),\; \phi\in[0,2\pi)\\
z &= z
\end{align*}

\begin{ex}
	A vector in rectangular has an expression $(1,1,1)$, then if we use cylindrical coordinate to express it, then
	\begin{align*}
	r &= \sqrt{1^2+1^2} = \sqrt{2}\\
	\phi &= \arctan(1/1) = \frac{\pi}{4}\\
	z &= 1
	\end{align*}
	
	Thus the representation of this vector in cylindrical system is $(\sqrt{3},\frac{\pi}{4},1)$. 
\end{ex}


\section{Spherical coordinate system}
\begin{center}
	\includegraphics*[width=7cm]{spherical-coordinates.png}
\end{center}
The representation of a vector in spherical coordinate system is $(\rho,\theta,\phi)$. Where $\rho$ is the length of the vector, $\theta$ the polar angle, and $\phi$ the azimuth angle. The transformation between rectangular coordinate and spherical coordinate is given by
\begin{align*}
x &= \rho\sin\theta\cos\phi\\
y &= \rho\sin\theta\sin\phi\\
z &= \rho\cos\theta
\end{align*}

or 

\begin{align*}
\rho &= \sqrt{x^2+y^2+z^2},\;\rho\in[0,\infty)\\
\theta &= \arccos(z/\rho)\\
\phi &= \arctan(y/x) 
\end{align*}


\begin{ex}
	A vector in spherical coordinate has a representation of $(4,\pi/3,0)$. Then its representation in rectangular coordinate will be given by
	\begin{align*}
	x &= 4\sin(\pi/3)\cos(0) = 2\sqrt{3}\\
	y &= 4\sin(\pi/3)\sin(0) = 0\\
	z &= 4\cos(\pi/3) = 2
	\end{align*}
	
	Thus its representation in rectangular coordinate is $(2\sqrt{3},0,2)$.
\end{ex}

\end{document}

