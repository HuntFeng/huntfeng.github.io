\documentclass{article}

\usepackage{amsmath}

\usepackage{amssymb}

\usepackage{amsthm}

\usepackage{graphicx}



\newtheorem{definition}{Definition}

\newtheorem{property}{Property}

\newtheorem{ex}{Example}

\newtheorem{thm}{Theorem}

\newtheorem{lemma}{Lemma}

\newtheorem{prop}{Proposition}

\newtheorem*{solution}{Solution}



\begin{document}
\section{Total Differentiation and Approximation}
\begin{definition}
	The \textbf{total derivative} of a function $f(x,y)$ at point $(x_0,y_0)$ is given by
	\[ df = \partial_x f(x_0,y_0) dx + \partial_y f(x_0,y_0) dy \]
	
	In general, for functions of n variables $f(\vec{x})$ where $\vec{x}=(x_1,x_2,\cdots,x_n)$, the total differential at point $\vec{x}_0$
	\[ df = \sum_{i=1}^n \partial_{x_i} f(\vec{x}_0) dx_i \]
\end{definition}

Note: Just think of this total derivative is just like the differentiation in first year calculus.

\begin{ex}
	Find the Total change of the area of a rectangle with length 10m and width 20m if the change in length is 0.5m and the change in width is 0.1m.
	\begin{solution}
		Let the area of the rectangle be $A(l,w) = lw$.
		\begin{align*}
		\Delta A &\approx \partial_l A\Delta l + \partial_w A \Delta w\\
		&= w\Delta l + l\Delta w\\
		&= 20\times 0.5 + 10\times 0.1\\
		&= 11 m^2 
		\end{align*}  
	\end{solution}  
\end{ex}

\section{Tangent Plane}
\begin{prop}
	Whenever there is a surface $z=f(x,y)$, the tangent plane that at point $(x_0,y_0,z_0)$ can be given by
\[ (z-z_0) = \partial_xf(x_0,y_0)(x-x_0) + \partial_yf(x_0,y_0)(y-y_0) \]
The normal vector of this plane is $(\partial_xf(x_0,y_0),\partial_yf(x_0,y_0),-1)$.
\end{prop}

It is easy to see that why this is true. As $(x,y)\to (x_0,y_0)$, the entire equation above comes back to its origin, total differential of the function at point $(x_0,y_0)$.
\[ dz =  \partial_xf(x_0,y_0)dx+ \partial_yf(x_0,y_0)dx \] 


\begin{ex}
	Find the tangent plane of the sphere $x^2+y^2+z^2=1$ at point (0,0,1).
	\begin{solution}
		First we convert the surface equation to the form $z=f(x,y)$. Since we only care about the north pole (0,0,1), so we just need to take the upper sphere
		\[ z= \sqrt{1-x^2-y^2} \]
		
		Thus the tangent plane at (0,0,1) would be
		\begin{align*}
		(z-z_0) &= \partial_xf(x_0,y_0)(x-x_0) + \partial_yf(x_0,y_0)(y-y_0)\\
		z-1 &= \left.\frac{-x}{\sqrt{1-x^2-y^2}}\right|_{(0,0)} (x-0) 
		+\left.\frac{-y}{\sqrt{1-x^2-y^2}}\right|_{(0,0)}  (y-0)\\
		z-1 &= 0 
		\end{align*}
		
		Thus tangent plane is just simply $z=1$. Easy to verify this. 
	\end{solution}
\end{ex}



\end{document}

